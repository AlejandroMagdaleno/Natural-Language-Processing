\begin{quote}
You are given a training set of 30 numbers that consists of 21 zeros and 1 each of the other digits 1-9. Now we see the following test set: 0 0 0 0 0 3 0 0 0 0. What is the unigram perplexity?
\end{quote}
\begin{table}[h!]
\centering
\begin{tabular}{|| c c c c c c c c c c |}
\hline
 0 & 1 & 2 & 3 & 4 & 5 & 6 & 7 & 8 & 9\\
 \hline
 \hline
 21 & 1 & 1 & 1 & 1 & 1 & 1 & 1 & 1 & 1\\
 \hline
\end{tabular}
\caption{Normalizing by Unigrams}

\begin{tabular}{|| c c c c c c c c c c |}
\hline
 0 & 1 & 2 & 3 & 4 & 5 & 6 & 7 & 8 & 9\\
 \hline
 \hline
 .7 & .03 & .03 & .03 & .03 & .03 & .03 & .03 & .03 & .03\\
 \hline
\end{tabular}
\caption{Computing Probabilities}
\end{table}
\begin{equation}
Unigram\; Perplexity = .7 * (.03 * .03 * .03 * .03 * .03 * .03 * .03 * .03 * .03) = 2.1
\end{equation}
